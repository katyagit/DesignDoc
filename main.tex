\documentclass{article}
\usepackage{graphicx} % Required for inserting images
\usepackage[T2A]{fontenc}

\title{Диздок}
\author{}
\date{Ноябрь 2024}

\begin{document}

\maketitle

\tableofcontents

\section{Введение}
Над созданием игры работали 6 человек: \\
Екатерина Наумова и Анастасия Веренинова, отвечающие за визуальную часть
проекта,\\
Егор Косых, Андрей Булычев и Иван Гайфиев, ответственные за разработку кода,\\
и Анна Малянова, оказывающая поддержку остальным участникам.

Идея сундука-обманки вдохновлена Мимиком из "Подземелье Вкусностей" Рёко Куи, монстром который маскировался под сундуки с золотом.

\section{Концепция}

\subsection{Введение}
Данная игра погружает пользователя в мир тёмного фэнтези с атмосферой 
средневековья, где его главной целью является исследование обширных 
и опасных локаций, сражения с боссами и, в конечном счёте, разгадка 
тайны игрового мира, в котором он оказался. \par Игра ориентирована на любителей 
фэнтези и мрачного лора в возрасте от 12 лет и старше. \par Одной из особенностей 
геймплея являются уникальные механики и событийная система, которая не 
даёт игроку заскучать благодаря случайным событиям, происходящим с его 
персонажем. \par Игра предназначена для запуска на персональных компьютерах 
с операционными системами Windows или macOS. \par Предпосылкой для её создания стали 
популярность жанра и любовь авторов к разработке увлекательных игр.

\subsection{Жанр и аудитория}
Игра представляет собой 2D-соулслайк в средневековом дарк-фэнтези стиле. Это жанр, который акцентирует внимание на сложных боях, исследовании обширных и атмосферных локаций. Игроки будут сталкиваться с различными врагами, включая монстров и боссов, требующими стратегического подхода и мастерства в бою. Элементы RPG, такие как прокачка персонажа, сбор предметов и взаимодействие с окружающим миром, придают глубину игровому процессу.\\[2mm] \parЦелевая аудитория игры — игроки в возрасте от 12 до 18 лет. Эта группа включает как опытных геймеров, знакомых с жанром соулслайков, так и новых игроков, ищущих интересный сюжет.\\[2mm] \parИгра будет позиционироваться как сложный и атмосферный проект, который предлагает игрокам уникальный опыт в исследовании мрачного фэнтезийного мира.  \\[2mm] \parПоэтому, игра будет не только развлекательной, но и предоставит игрокам возможность погрузиться в атмосферный и сложный мир дарк-фэнтези, где каждая локация отличается завораживающим окружением.

\subsection{Основные особенности игры}
То, что выделяет эту игру на фоне других в данном жанре - уникальные механики и система событий в игровом мире, которые не дадут игроку стоять на месте и расслабляться. События могут происходить случайно либо после определенного взаимодействия игрока с окружением. \\[2mm] \parТуман - случайное событие в локациях открытой местности (не внутри помещений), блокирует видимость игрока если рядом нет источника света. Нахождение в тумане более 5 секунд вызывает галлюцинации, иммитирующие звуки врагов-монстров и горящие глаза. Туман наносит переодический урон, иллюзиорным врагам нельзя нанеси урон. Чтобы избежать вред события, необхдимо найти источник света или иметь его экиированным. \\[2mm] \parСундуки-обманки - предмет расположенный на локации, внешне напоминает обычный сундук, но отличнается пятном на крышке. При нахождении рядом с данным предметом  некоторый промежуток времени (без взаимодействия) сундук-обманка показывает зубы, что позволяет вычислить его. При взаимодействии с сундуком-обманкой игрок теряет определенное количество здоровье и имеет риск потери случайного предмета из экипировки. \\[2mm] 

\subsection{Описание игры}

\subsection{Предпосылки создания}
Хотелось бы начать с описания факта популярности игр в современном мире. Люди проводят так свободное время, отдыхают, находят друзей в онлайн играх. Также как и большинство, участники этого проекта являются фанатами игр. Именно поэтому мы решили создать игру, которая будет интересна именно нам и также другим любителям игр. 

Как уже было замечено, наша игра представлена в стиле souls-like в 2D формате. В современном мире этот стиль занимает высокие позиции по популярности, поэтому нас заинтересовало сделать что-то новое именно в этом жанре. Идеально отлаженный игровой процесс, сочетающийся с высокой сложностью, внутриигровое социальное взаимодействие, огромный потенциал для обсуждения, таинственный и захватывающий мир, который требует тщательного изучения и исследования - все это основные черты нашей игры. 

Безусловно наш проект может быть похож на другие в этом жанре, хотя и не имеет привязке к уже созданным вселенным из книг и фильмов, но также включает в себя много уникальных компонентов и идей (локации, тайны для разгадки и многое другое). 

\subsection{Платформа}
\subsubsection{OC Windows}
Минимальные:
\begin{itemize}
    \item 64-разрядные процессор и операционная система
    \item ОС: Windows 10
    \item Процессор: Intel Core 2 Duo E8400 or AMD Phenom II X2 550
    \item Оперативная память: 4 GB ОЗУ
    \item Видеокарта: NVIDIA GeForce GT 520, 1 GB or AMD Radeon HD 7470, 1 GB or Intel HD Graphics 4400
    \item Место на диске: 4 GB
    \item Звуковая карта: любая встроенная
    \item Управление: клавиатура, геймпад
    \item CD-ROM привод: не требуется
\end{itemize}

Рекомендованные:
\begin{itemize}
    \item 64-разрядные процессор и операционная система
    \item ОС: Windows 10
    \item Процессор: Intel Core i3-550 or AMD FX-4100
    \item Оперативная память: 6 GB ОЗУ
    \item Видеокарта: NVIDIA GeForce GT 710, 1 GB or AMD Radeon R7 240, 1 GB or Intel HD Graphics 530
    \item Место на диске: 4 GB
    \item Звуковая карта: любая встроенная
    \item Управление: клавиатура, геймпад
    \item CD-ROM привод: не требуется
\end{itemize}

\subsubsection{MacOS}
Минимальные:
\begin{itemize}
    \item ОС: MacOS Sierra - 10.12.6
    \item Процессор: 2.7 GHz Intel Core i5
    \item Оперативная память: 4 GB ОЗУ
    \item Видеокарта: Nvidia GeForce GT 640M
    \item Место на диске: 4 GB
    \item Звуковая карта: любая встроенная
    \item Управление: клавиатура, геймпад
    \item CD-ROM привод: не требуется
\end{itemize}

Рекомендованные:
\begin{itemize}
    \item ОС: Mac OS 10.13 (64bit)
    \item Процессор: Intel Core i5
    \item Оперативная память: 6 GB ОЗУ
    \item Видеокарта: GeForce GTX 570
    \item Место на диске: 4 GB
    \item Звуковая карта: любая встроенная
    \item Управление: клавиатура, геймпад
    \item CD-ROM привод: не требуется

\end{itemize}

\section{Функциональная спецификация}

\subsection{Принципы игры}

\subsubsection{Суть игрового процесса}

\subsubsection{Ход игры и сюжет}

\subsection{Физическая модель}

\subsection{Персонаж игрока}

\subsection{Элементы игры}

\subsection{«Искусственный интеллект»}

\subsection{Многопользовательский режим}

\subsection{Интерфейс пользователя}

\subsubsection{Блок-схема}

\subsubsection{Функциональное описание и управление}

\subsubsection{Объекты интерфейса пользователя}

\subsection{Графика и видео}

\subsubsection{Общее описание}

\subsubsection{Двумерная графика и анимация}

\subsubsection{Трехмерная графика и анимация}

\subsubsection{Анимационные вставки}

\subsection{Звуки и музыка}

\subsubsection{Общее описание}

\subsubsection{Звук и звуковые эффекты}

\subsubsection{Музыка}

\subsection{Описание уровней}

\subsubsection{Общее описание дизайна уровней}

\subsubsection{Диаграмма взаимного расположения уровней}

\subsubsection{График введения новых объектов}

\section{Контакты}

\newpage

\title 

\end{document}
